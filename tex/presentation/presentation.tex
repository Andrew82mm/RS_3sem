\documentclass[9pt]{beamer}

\usepackage[T2A]{fontenc}
\usepackage[utf8]{inputenc}
\usepackage[russian]{babel}
\usepackage{amsmath, amssymb}
\usepackage{booktabs}
\usepackage{xcolor}

\usetheme{CambridgeUS}

% Уменьшенный footline
\setbeamertemplate{footline}{
	\leavevmode%
	\hbox{%
		\begin{beamercolorbox}[wd=.333333\paperwidth,ht=1.5ex,dp=0.75ex,center]{author in head/foot}%
			\usebeamerfont{author in head/foot}\tiny\insertshortauthor
		\end{beamercolorbox}%
		\begin{beamercolorbox}[wd=.333333\paperwidth,ht=1.5ex,dp=0.75ex,center]{title in head/foot}%
			\usebeamerfont{title in head/foot}\tiny\insertshorttitle
		\end{beamercolorbox}%
		\begin{beamercolorbox}[wd=.333333\paperwidth,ht=1.5ex,dp=0.75ex,right]{date in head/foot}%
			\usebeamerfont{date in head/foot}\tiny\insertframenumber{} / \inserttotalframenumber\hspace*{2ex} 
	\end{beamercolorbox}}%
	\vskip0pt%
}

\title[Применение KNN]{Применение алгоритмов K-ближайших соседей в коллаборативных рекомендательных системах}
\author[А. Сергиенко]{Сергиенко Андрей}
\date{2025}

\begin{document}
	
	% Титульный слайд
	\begin{frame}[plain]
		\centering
		
		{\small САНКТ-ПЕТЕРБУРГСКИЙ ГОСУДАРСТВЕННЫЙ УНИВЕРСИТЕТ \\[2mm]
			Искусственный интеллект и наука о данных}
		
		\vspace{1 cm}
		
		{\Large\bfseries
			Применение алгоритмов K-ближайших соседей \\[2mm]
			в коллаборативных рекомендательных системах
		}
		
		\vspace{5 mm}
		
		{\large Отчёт о прохождении учебной (ознакомительной) практики}
		
		\vfill
		
		\vspace{7 mm}
		
		\raggedright
		{\normalsize
			\textbf{Автор:} \\
			Сергиенко Андрей \\[4mm]
			
			\textbf{Научный руководитель:} \\
			старший преподаватель кафедры системного программирования \\
			Юрий Александрович Андреев
		}
		
		\vspace{1.2cm}
		
		\centering
		Санкт-Петербург, 2025
	\end{frame}
	
	\begin{frame}{Постановка задачи}
		\begin{block}{Цель работы}
			Исследовать эффективность различных алгоритмов K-ближайших соседей (KNN) для реализации коллаборативной рекомендательной системы.
		\end{block}
		
		\begin{block}{Задачи}
			\begin{itemize}
				\item Изучить принципы коллаборативной фильтрации и KNN
				\item Реализовать и сравнить 4 метода поиска:
				\begin{itemize}
					\item \textbf{Exact KNN} (scikit-learn) — эталонный метод
					\item \textbf{Annoy}  — случайные деревья
					\item \textbf{FAISS}  — кластеризация IVF
					\item \textbf{HNSW} — иерархические графы
				\end{itemize}
				\item Провести измерения производительности, точности и памяти
				\item Сформулировать рекомендации по применению
			\end{itemize}
		\end{block}
	\end{frame}
	
	\begin{frame}{Методология}
		\begin{block}{Датасет MovieLens (32M)}
			\begin{itemize}
				\item \textbf{Фильтрация данных:}
				\begin{itemize}
					\item Пользователи: > 50 оценок → \textcolor{blue}{126 588 чел.}
					\item Фильмы: > 10 оценок → \textcolor{blue}{30 521 шт.}
				\end{itemize}
				\item \textbf{Матрица:} $126\,588 \times 30\,521$ (\textcolor{red}{99.24\% разреженность})
			\end{itemize}
		\end{block}
		
		\begin{block}{Снижение размерности через TruncatedSVD}
			\begin{itemize}
				\item \textbf{Цель:} Переход от разреженного пространства к плотным эмбеддингам
				\item \textbf{Размерность:} 128 латентных компонентов
				\item \textbf{Результат:} Объяснено 42.07\% дисперсии, сжатие в 96 раз
			\end{itemize}
		\end{block}
		
		\vspace{0.3cm}
		\centering
		\small{$R \approx U_k \Sigma_k V_k^T$ — сохраняем только главные компоненты}
	\end{frame}
	
	\begin{frame}{Архитектура измерений}
		\begin{block}{Изолированные процессы для точности}
			\begin{columns}[T]
				\begin{column}{0.48\textwidth}
					\textbf{Orchestrator:}
					\begin{itemize}
						\item Подготовка данных
						\item Генерация ground truth
						\item Запуск worker-ов
						\item Агрегация результатов
					\end{itemize}
				\end{column}
				\begin{column}{0.48\textwidth}
					\textbf{Worker (subprocess):}
					\begin{itemize}
						\item Чистая память
						\item Построение индекса
						\item Выполнение запросов
						\item Возврат метрик
					\end{itemize}
				\end{column}
			\end{columns}
		\end{block}
		
		\begin{block}{Метрики оценки}
			\begin{itemize}
				\item \textbf{Производительность:} Время построения, время запроса, память (psutil)
				\item \textbf{Точность:} Recall@20, Precision@20 (сравнение с Exact KNN)
				\item \textbf{Тестирование:} 100 случайных запросов (seed=42)
			\end{itemize}
		\end{block}
	\end{frame}
	
	\begin{frame}{Оптимизация гиперпараметров}
	\begin{block}{Композитная метрика Grid Search}
		\[
		\text{Score} = 0.4 \cdot \text{Recall@20} - 0.6 \cdot \ln(1 + \text{QueryTime}_{\text{ms}})
		\]
		\small{Баланс точности (40\%) и скорости (60\%) для production}
	\end{block}
	
	\begin{center}
		\begin{tabular}{lll}
			\toprule
			\textbf{Метод} & \textbf{Оптимальные параметры} & \textbf{Score} \\
			\midrule
			Annoy & n\_trees=50 & 0.199 \\
			FAISS & nlist=1600 & 0.291 \\
			HNSW & ef\_construction=200, M=16 & \textbf{0.327} \\
			\bottomrule
		\end{tabular}
	\end{center}
	
	\vspace{0.3cm}
	\small{HNSW показал наилучший компромисс при минимальных параметрах}
	\end{frame}
	
	\begin{frame}{Результаты: Точность}
		\begin{center}
			\begin{tabular}{lccc}
				\toprule
				\textbf{Метод} & \textbf{Recall@20} & \textbf{Precision@20} & \textbf{Потеря точности} \\
				\midrule
				Exact KNN & 1.000 & 1.000 & --- \\
				HNSW & 0.973 & 0.973 & 2.7\% \\
				FAISS & 0.939 & 0.939 & 6.1\% \\
				Annoy & 0.760 & 0.760 & 24.0\% \\
				\bottomrule
			\end{tabular}
		\end{center}
		
		\vspace{0.5cm}
		
		\begin{alertblock}{Ключевой вывод}
			\textbf{HNSW} достигает почти эталонной точности (97.3\%), что критично для персональных рекомендаций, где каждый процент влияет на вовлечённость пользователей.
		\end{alertblock}
	\end{frame}
	
	\begin{frame}{Результаты: Производительность}
		\begin{center}
			\begin{tabular}{lcc}
				\toprule
				\textbf{Метод} & \textbf{Время запроса (user)} & \textbf{Ускорение} \\
				\midrule
				Exact KNN & 139.59 мс & 1× \\
				Annoy & 0.19 мс & 734× \\
				FAISS & 0.18 мс & 775× \\
				HNSW & 0.12 мс & \textbf{1163×} \\
				\bottomrule
			\end{tabular}
		\end{center}
		
		\vspace{0.5cm}
		
		\begin{block}{Время построения индекса (user-based)}
			\begin{itemize}
				\item Exact KNN: 0.006 с (без структур)
				\item Annoy: 2.78 с
				\item HNSW: 7.27 с
				\item FAISS: 8.54 с
			\end{itemize}
		\end{block}
		
		\small{\textit{Замедление построения окупается тысячекратным ускорением запросов}}
	\end{frame}
	
	\begin{frame}{Результаты: Потребление памяти}
		\begin{center}
			\begin{tabular}{lcc}
				\toprule
				\textbf{Метод} & \textbf{Peak Memory} & \textbf{Относительно} \\
				\midrule
				Exact KNN & 0.00 МБ & (базовая линия) \\
				HNSW & 121.05 МБ & 1.0× \\
				FAISS & 131.54 МБ & 1.1× \\
				Annoy & 314.74 МБ & 2.6× \\
				\bottomrule
			\end{tabular}
		\end{center}
		
		\vspace{0.5cm}
		
		\begin{alertblock}{Баланс скорость—точность—память}
			\textbf{HNSW} лидирует по всем критериям:
			\begin{itemize}
				\item Максимальная скорость запросов (0.12 мс)
				\item Высочайшая точность среди ANN (97.3\%)
				\item Минимальное потребление памяти (121 МБ)
			\end{itemize}
		\end{alertblock}
	\end{frame}
	
	\begin{frame}{Итоговое сравнение}
	\begin{center}
		\includegraphics[width=1.0\textwidth]{knn_comparison.png}
	\end{center}
	\end{frame}
	
	\begin{frame}{Выводы и рекомендации}
		\begin{block}{1. Exact KNN — Эталон для тестирования}
			Офлайн-аналитика, малые датасеты (< 10K объектов)
		\end{block}
		
		\begin{block}{2. Annoy — Максимальная скорость}
			Когда допустима потеря 24\% точности: предварительная фильтрация, low-priority рекомендации
		\end{block}
		
		\begin{block}{3. FAISS — Масштабируемость + GPU}
			Большие системы (> 1M объектов), особенно с GPU-ускорением
		\end{block}
		
		\begin{block}{4. \textcolor{blue}{HNSW — Production-стандарт}}
			\textbf{Оптимальный выбор:} Почти эталонная точность (97.3\%), максимальная скорость (1163× быстрее), минимальная память (121 МБ). Рекомендован для высоконагруженных систем реального времени.
		\end{block}
	\end{frame}
	
	\begin{frame}
		\begin{center}
			\Huge{Спасибо за внимание!}
			
			\vspace{1.5cm}
			
			\large{Вопросы?}
			
			\vspace{0.5 cm}
			
			\small{Код и результаты доступны в репозитории проекта}
			
			\vspace{0.2 cm}
			
			\includegraphics[width=0.3\textwidth]{qr.png}
		\end{center}
	\end{frame}
	
\end{document}